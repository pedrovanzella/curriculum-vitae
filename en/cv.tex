\documentclass{tccv}
\usepackage[english]{babel}
\usepackage[utf8]{inputenc}  

\begin{document}

\part{Pedro Vanzella}

\section{Experience}

\begin{eventlist}

\item{August 2015 -- Present}
     {Python, C, JavaScript, AngularJS}
     {CWI Software - Renner}

Desenvolvimento de um sistema de terminais de auto-atendimento, em plataforma
Linux. Um middleware em Python se comunica com um back-end Java, hardware e
bibliotecas em C que processam pagamentos, e é exposto em JavaScript por um
browser Webkit, em C. Front-end em AngularJS se comunica com as funções expostas
pelo browser.

\item{May 2015 -- August 2015}
     {Python, Flask, MongoDB}
     {CWI Software - Terra}

Sistema de Video On Demand (VOD) do TerraTV. Manutenção de código legado e inclusão
de novos recursos em uma aplicação em Flask.

\item{January 2015 -- May 2015}
     {Java, Spring, Hibernate, Postgres}
     {SystemHaus}

Manutenção de código legado e inclusão de novos recursos em um sistema de
controle de cadeia de produção. Base de código com centenas de milhares de
linhas de código, mantida há mais de dez anos.

\item{July 2014 -- December 2014}
     {Ruby, Rails, MongoDB}
     {Fator7 - StarlightLED}

Desenvolvimento de sistema de controle para Internet of Things (IoT). Uma
aplicação Rails, na nuvem, controla painéis de iluminação pública.

\item{July 2013 -- December 2013}
     {Ruby, Rails, Postgres}
     {TopTal - USEED}

Desenvolvimento de um sistema de \textit{crowdfunding} em Rails, em um time
100\% remoto e internacional.

\item{November 2012 -- July 2013}
     {Ruby, Rails, Postgres}
     {UFRGS - NetMetric - Vivo}
 
Desenvolvimento de um sistema em Rails para o monitoramento de sondas de rede celular.

\item{February 2012 -- November 2012}
     {Python, Linux}
     {UFRGS - CPD}

Desenvolvimento e manutenção de sistema de controle de acesso e registro de
estações da universidade.

\end{eventlist}

\personal
    [www.pedrovanzella.com]
    {Rua João Cândido, 500\newline Sapucaia do Sul (RS) - Brazil}
    {+55 (51) 34743130}
    {pedro@pedrovanzella.com}

\section{Education}

\begin{yearlist}

\item[B.A.]{2013 -- 2016}
     {Computer Science}
     {PUCRS}

\item[B.A.]{2009 -- 2012}
     {Computer Engineering}
     {UFRGS}

\end{yearlist}

\section{Personal Projects}

\begin{yearlist}
  
\item{2015}
     {chickencross (\href{https://github.com/vantas/chickencross}{github.com/vantas/chickencross})}
     {Clone de Frogger em C++}

\item{2014}
     {dolly (\href{https://github.com/pedrovanzella/Ball}{github.com/pedrovanzella/Ball})}
     {Game engine em C++}

\item{2014}
     {Flask-DODDNS (\href{https://github.com/pedrovanzella/Flask-DODDNS}{github.com/pedrovanzella/Flask-DODDNS})}
     {Aplicação em Flask para prover DNS dinâmico para roteadores}

\end{yearlist}

\section{Communication Skills}

\begin{factlist}
\item{English}{Advanced, CEF C2}
\item{Portuguese}{Native, CEF C2}
\item{Spanish}{Intermediate, CEF B1}
\end{factlist}

\section{Technical Skills}

\begin{factlist}

\item{Programming Languages}
     {Python, Ruby, JavaScript, C, C++, Java}
     
\item{Frameworks}
     {Ruby on Rails, Django, Flask, AngularJS}
     
\item{Methodologies}
     {Scrum, Kanban, XP, TDD, BDD}

\item{Tools}
     {PostgreSQL, Oracle, Git, MongoDB}

\item{Operating Systems}
     {Linux, FreeBSD, Mac OSX}

\end{factlist}

\end{document}
