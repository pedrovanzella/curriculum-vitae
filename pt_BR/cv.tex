\documentclass{tccv}
\usepackage[brazil]{babel}
\usepackage[utf8]{inputenc}  

\begin{document}

\part{Pedro Vanzella}

\section{Experiência}

\begin{eventlist}

\item{Agosto 2015 -- Presente}
     {Python, C, JavaScript, AngularJS}
     {CWI Software - Renner}

Desenvolvimento de um sistema de terminais de auto-atendimento, em plataforma
Linux. Um middleware em Python se comunica com um back-end Java, hardware e
bibliotecas em C que processam pagamentos, e é exposto em JavaScript por um
browser Webkit, em C. Front-end em AngularJS se comunica com as funções expostas
pelo browser.

\item{Maio 2015 -- Agosto 2015}
     {Python, Flask, MongoDB}
     {CWI Software - Terra}

Sistema de Video On Demand (VOD) do TerraTV. Manutenção de código legado e inclusão
de novos recursos em uma aplicação em Flask.

\item{Janeiro 2015 -- Maio 2015}
     {Java, Spring, Hibernate, Postgres}
     {SystemHaus}

Manutenção de código legado e inclusão de novos recursos em um sistema de
controle de cadeia de produção. Base de código com centenas de milhares de
linhas de código, mantida há mais de dez anos.

\item{Julho 2014 -- Dezembro 2014}
     {Ruby, Rails, MongoDB}
     {Fator7 - StarlightLED}

Desenvolvimento de sistema de controle para Internet of Things (IoT). Uma
aplicação Rails, na nuvem, controla painéis de iluminação pública.

\item{Julho 2013 -- Dezembro 2013}
     {Ruby, Rails, Postgres}
     {TopTal - USEED}

Desenvolvimento de um sistema de \textit{crowdfunding} em Rails, em um time
100\% remoto e internacional.

\item{Novembro 2012 -- Julho 2013}
     {Ruby, Rails, Postgres}
     {UFRGS - NetMetric - Vivo}
 
Desenvolvimento de um sistema em Rails para o monitoramento de sondas de rede celular.

\item{Fevereiro 2012 -- Novembro 2012}
     {Python, Linux}
     {UFRGS - CPD}

Desenvolvimento e manutenção de sistema de controle de acesso e registro de
estações da universidade.

\end{eventlist}

\personal
    [www.pedrovanzella.com.br]
    {Rua João Cândido, 500\newline Sapucaia do Sul (RS)}
    {(51) 34743130}
    {pedro@pedrovanzella.com}

\section{Educação}

\begin{yearlist}

\item[Em andamento]{2013 -- 2016}
     {Ciência da Computação}
     {PUCRS}

\item{2009 -- 2012}
     {Engenharia da Computação}
     {UFRGS}

\end{yearlist}

\section{Projetos Públicos}

\begin{yearlist}
  
\item{2015}
     {chickencross (\href{https://github.com/vantas/chickencross}{github.com/vantas/chickencross})}
     {Clone de Frogger em C++}

\item{2014}
     {dolly (\href{https://github.com/pedrovanzella/Ball}{github.com/pedrovanzella/Ball})}
     {Game engine em C++}

\item{2014}
     {Flask-DODDNS (\href{https://github.com/pedrovanzella/Flask-DODDNS}{github.com/pedrovanzella/Flask-DODDNS})}
     {Aplicação em Flask para prover DNS dinâmico para roteadores}

\end{yearlist}

\section{Línguas}

\begin{factlist}
\item{Inglês}{Avançado, nativo}
\end{factlist}

\section{Habilidades Técnicas}

\begin{factlist}

\item{Linguagens}
     {Python, Ruby, JavaScript, C, C++, Java}
     
\item{Frameworks}
     {Ruby on Rails, Django, Flask, AngularJS}
     
\item{Metodologias}
     {Scrum, Kanban, XP, TDD, BDD}

\item{Ferramentas}
     {PostgreSQL, Oracle, Git, MongoDB}

\item{Sistemas Operacionais}
     {Linux, FreeBSD, Mac OSX}

\end{factlist}

\end{document}
