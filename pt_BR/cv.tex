\documentclass{tccv}
\usepackage[brazil]{babel}
\usepackage[utf8]{inputenc}  

\begin{document}

\part{Pedro Vanzella}

\section{Experiência}

\begin{eventlist}

\item{July 2007 -- Present}
     {eN.TiDi software, Travagliato}
     {Management and development}

Software development for the industrial automation sector: configuration
front-end in C with interface based on \href{http://www.gtk.org/}{GTK+},
web applications and sites on LAMP platforms grounded on the
\href{http://www.silverstripe.org/}{SilverStripe} framework,
supervisor programs in LabVIEW and remote system in
\href{http://www.lua.org/}{Lua} on GNU/Linux systems.

\item{January 2002 -- June 2004}
     {TEMA s.r.l., Travagliato}
     {PLC Omron development}

Development and testing of automatic and semiautomatic machines for
ribbon winding on spools and rolls (safety belts, hook-and-loop tapes,
elastic ribbon and bindings). Designing of electrical schematics on
2D cad and user manuals drafting.

\item{October 1998 -- November 2001}
     {TWINS s.r.l., Sarezzo}
     {PC and PLC Siemens development}

Programming, installation and trial of transfer machines for assembly,
adjustment and testing of gas taps. Development and installation of PC
based semiautomatic test stands for pneumatic and hydraulic leakage
tests on valves, gas regulators, electrovalves, tanks and others.

\item{September 1996 -- September 1998}
     {Elettronica EFFE-GI s.n.c., Cazzago}
     {PC and PLC Hitachi development}

Development of automatic machines in general, here included the design
of electrical, pneumatic and hydraulic schematics on 2D cad and the PLC
programming. Implementation of configuration and logging front-end using
RS232 serial communications between PC and PLC-based transfer machines
in BASIC, Pascal and C.

\item{January 1994 -- June 1996}
     {Seven Diesel s.p.a., Rovato}
     {Engineering department director}

Planning of nozzles for diesel engines and organization of production.
Upgraded the management software by implementing a system for data
handling and automatic drawing generation (in DXF) developed on the
SuperBase 95 RAD tool.

\end{eventlist}

\personal
    [www.pedrovanzella.com.br]
    {Rua João Cândido, 500\newline Sapucaia do Sul (RS)}
    {(51) 34743130}
    {pedro@pedrovanzella.com}

\section{Educação}

\begin{yearlist}

\item[High school diploma]{1988 -- 1992}
     {Informatic engineer}
     {ITIS Castelli, Brescia}

\item{1987 -- 1988}
     {Classical gymnasium}
     {Seminario vescovile, Cremona}

\end{yearlist}

\section{Projetos Públicos}

\begin{yearlist}

\item{2012}
     {ntdisp (\href{http://ntdisp.entidi.com/}{ntdisp.entidi.com})}
     {Embedded devices programmer}

\item{2007}
     {tip (\href{http://tip.entidi.com/}{tip.entidi.com})}
     {PHP framework based on PEAR}

\item{2006}
     {adg (\href{http://adg.entidi.com/}{adg.entidi.com})}
     {Automatic drawing generation}

\item{2006}
     {gtk2panel (\href{http://gtk2panel.entidi.com/}{gtk2panel.entidi.com})}
     {Top panel menu in GTK+2}

\item{2004}
     {ntd (\href{http://ntd.entidi.com/}{ntd.entidi.com})}
     {General purpose libraries}

\end{yearlist}

\section{Línguas}

\begin{factlist}
\item{Italian}{Native speaker}
\item{English}{Oral: fair -- Written: good}
\item{Spanish}{Oral: good}
\end{factlist}

\section{Habilidades Técnicas}

\begin{factlist}

\item{Good level}
     {C, PHP, HTML, CSS, autotools, git, gcc, GTK+, GObject, shell,
      MS-DOS, Linux, ladder, G-Code}

\item{Intermediate}
     {Lua, \LaTeX, MySQL, VBA, cuBasic, pascal, subversion, LabVIEW}

\item{Basic level}
     {Windows, FreeBSD, OpenIndiana, Postgres}

\end{factlist}

\end{document}
